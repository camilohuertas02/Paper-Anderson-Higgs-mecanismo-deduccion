\section{¿Qué problemas solucionó el Mecanismo?}

hablamos de los bosones w y z con masa

unificacion de electromagnetismo y debil en teoria electrodebil (¿que ocurre con la fuerza nuclear fuerte?, es necesaria mencionarla en el mecansimo?, si no, al menos podemos dar un tipo de pregunta abierta al lector para que se pregunte si la fuerza nuclear fuerte puede explicarse por principio de gauge.)

hablamos de acoplamientos de yukawa para explicar la amsa de los fermiones (electrones (leptones), quarks (bariones)).


\section{¿Qué problemas deja el Mecanismo?}

\begin{itemize}
	\item el problema de la jerarquia: masa del boson de higgs muy ligera comparada con la escala de planck. explicamos porqué esto es un problema.

	\item el problema de la cosntante cosmologica: la energia de vacio que nos dice el potencial de higgs (y su campo de higgs) es mayor a la energia oscuraque se observa (¿ por que?, esto es clave que profundicemos, porque se tiene la energia oscura en un santo grial).

	\item por que las masas son las que son.

	\item nada sobre la gravedad ni materia oscura.

\end{itemize}




