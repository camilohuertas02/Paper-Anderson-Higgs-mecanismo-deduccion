\section{Resumen}

Se presenta un marco conceptual para la comprensión del mecanismo de Higgs, diseñado con un enfoque pedagógico accesible para estudiantes de pregrado en física. El objetivo es dilucidar el origen de la masa de las partıculas elementales, conectándolo con los principios fundamentales de simetría en la teoría cuántica de campos.

La metodología parte de la distinción entre simetrías gauge globales y locales para introducir el concepto de ruptura espontánea de la simetría (RSS). Se utiliza el potencial de “sombrero mexicano” como arquetipo para ilustrar cómo un lagrangiano simétrico genera un estado de vacío asimétrico, lo que conduce a una rigidez de fase y a la aparición de bosones de Nambu-Goldstone, en concordancia con su teorema.

El resultado central es la elucidación del mecanismo de Higgs. Se demuestra que, al promover la simetría a una de tipo gauge local, los bosones de Goldstone –que constituirían una inconsistencia fenomenológica– son absorbidos por los campos de gauge, originalmente sin masa. Este proceso no solo dota de masa a los bosones vectoriales (W y Z), sino que también predice la existencia de una excitación escalar masiva residual: el bosón de Higgs.

En conclusión, este trabajo ofrece una construcción lógica y autocontenida que vincula la abstracción de las simetrías con el fenómeno físico de la masa, proveyendo una herramienta didáctica esencial para la apreciación de uno de los pilares del Modelo Estándar.
