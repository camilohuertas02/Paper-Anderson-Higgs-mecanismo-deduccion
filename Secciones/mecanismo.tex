\section{La electrodinamica cuantica (QED)}

Fue la primera teoria que este enfoque del Principio de Gauge dio grandes frutos

Hablamos de QTF

\subsection{Campo del electron}

Introducimos el campo fermionico, la estadistica de fermi-dirac, y el campo de dirac, un campo fermionico para describir particulas de spin -1/2. Introducimos la ecuacion de dirac, usemos notacion slash de Feynman.

\subsection{grupo U(1)}

introduccion el grupo U(1), y mostramos que la cantidad invariante es la carga electrica (con simetria global), luego la promovemos localmente, para mostrar como aparece un campo de gauge(de fuerza) que interpretamos como el campo electromagnetico. mostramos la aparicion del un boson sin masa, el foton.

\section{Teoria electrodebil, el intento con SU(2)xU(1)}

El siguiente gran paso fue aplicar lo mismo a la teoria electrodebil SU(2)xU(1). Obtenemos bosones W y z sin masa, pero los experimentos muestran que tienen masa (hablamos de la relacion que tiene el alcance de una fuerza con que tan masivo es su boson).


\section{ruptura espontanea de la simetria}

presentamos el potencial de sobrero mexicano.

\subsection{ruptura de la simetria globalmente}

mostrar como se hace y que resultados se obtiene.

\subsection{ruptura de la simetria localmente}

mostramos como se hace y que resultados se obtiene.

\subsection{ideas de faltan}

encontrar la forma de incorporar los bosones de goldstone y hablar del tema de los grados de libertad, y todo eso.

llegamos a que los bosones W y Z ganan un grado de liberdad lo que les da masa.





